\chapter{Conclusion}
\label{ch:Conc}

This dissertation described the derivation and mathematical foundation of the Rayleigh Quotient Fixed Point method and its use to determine the alpha- and $k$-effective eigenvalue criticality problems in neutron transport. The Rayleigh Quotient Fixed Point method is a non-linear fixed point method that uses the proven primitivity property of the discretized neutron transport eigenvalue equations to find the positive angular flux eigenvector solution and corresponding eigenvalue. This dissertation discusses the application of the Rayleigh Quotient Fixed Point method to infinite media, one-dimensional slabs and sphere, as well as realistic two- and three-dimensional reactor models using the neutral particle transport code ARDRA \cite{hanebutte_ardra_1999}.

For alpha-eigenvalue problems, previous methods were limited to transport-based matrix method or criticality search iterative schemes. For transport-based matrix methods, the discretized matrices of the alpha-eigenvalue were formed and traditional eigenvalue solution methods were applied. These methods become increasingly expensive in memory and computational effort as problems increase in complexity. For criticality search iterative schemes, the alpha-eigenvalue is determined by relating the eigenvalue to another eigenvalue problem, the $k$-effective eigenvalue. These methods require multiple $k$-effective eigenvalue calculations to converge the alpha-eigenvalue, increasing the number of transport sweeps required to converge the alpha-eigenvalue. A substantial amount of iterations are spent determining an eigenvalue not of interest to the problem. Furthermore, there remain open questions as to how converged the proxy eigenvalue calculation must be to allow for convergence of the alpha-eigenvalue problem. These methods also historically have been applied to supercritical nuclear systems, being of limited use for subcritical problems that are become more of interest in the recent years. The Rayleigh Quotient Fixed Point for alpha-eigenvalue problems is an iterative method that uses an eigenvalue update that is optimal in the least squares sense. The Rayleigh Quotient Fixed Point method directly updates the alpha-eigenvalue, not requiring knowledge of any other eigenvalue, substantially reducing the number of iterations required for convergence. Additionally, it is able to solve subcritical. critical, and supercritical systems, providing a more general solution method to alpha-eigenvalue problems.

For $k$-effective problems, the traditional workhorse method, the power method, traditionally used a fission source update for the eigenvalue at each iteration. By using the Rayleigh Quotient Fixed Point method, the $k$-effective eigenvalue problem can be viewed as a non-linear fixed point method where an optimal update for the eigenvalue can be derived. In certain circumstances, this provides a reduction in transport sweeps necessary to converge the eigenvalue and eigenvector of interest.

Throughout the derivation of the Rayleigh Quotient Fixed Point method, the primitivity property of the discretized eigenvalue equations was used to guarantee the existence of a unique positive eigenvector corresponding to the spectral radius as stated in the Perron-Froebenius theory for primitive matrices. It was shown that for a one-dimensional slab geometry eigenvalue problem discretized with diamond differencing in space, discrete ordinates angular quadrature, and the multigroup-in-energy approximation, the discretized linear system is a primitive system with index of primitivity of two. This result is similar to that of Mokhtar-Kharroubi where it was found that the continuous neutron transport equation eigenvalue problem was \textit{positivity-improving} or primitive with index of primitivity of two \cite{mokhtar1997mathematical}.The existence of a unique eigenvector and a way to relate it to either the alpha- or $k$-effective eigenvalue provided a powerful tool to derive a fixed point method capable of determining the eigenvalue and the physical, positive angular flux eigenvector.

The Rayleigh Quotient Fixed Point method for alpha- and $k$-effective eigenvalues accurately determined the alpha- and $k$-effective eigenvalues of various analytic, infinite-medium problems by Betzler \cite{Betzler2014Alpha}. It was also demonstrated the failure of the eigenvalue problem when the conditions of irreducibility and primitivity failed to exist and how these problems were unphysical in most cases. Next, the method was validated for one-dimensional slabs and spheres and compared to the Green's Function Method of Kornreich and Parsons \cite{kornreich_timeeigenvalue_2005}. The number of transport sweeps required by the Rayleigh Quotient Fixed Point method was compared to the standard solvers, showing that the RQFP method provided major reductions in the number of iterations required for convergence and the ability to converge problems where other methods failed. Next, realistic two-dimensional cylindrical problems and two- and three-dimensional fuel assembly benchmark problems were analyzed to show the general applicability of the RQFP method even when some of the assumptions made in deriving the method did not apply. Transport sweep comparisons were used to show the improvement the RQFP method provides over traditional eigenvalue solvers for both alpha- and $k$-effective eigenvalue problems. For realistic reactor problems with large amounts of materials, energies, and other heterogeneities, the RQFP method outperformed the standard eigensolvers.

