\chapter{Conclusion}
\label{ch:Conc}

This dissertation described the derivation and mathematical foundation of the Rayleigh Quotient Fixed Point method and its use to determine the alpha- and $k$-effective eigenvalue criticality problems in neutron transport. The Rayleigh Quotient Fixed Point method is a non-linear fixed-point method that uses the proven primitivity property of the discretized neutron transport eigenvalue equations to find the positive angular flux eigenvector solution and corresponding eigenvalue. This dissertation discusses the application of the Rayleigh Quotient Fixed Point method to infinite media, one-dimensional slabs and spheres, as well as realistic two- and three-dimensional reactor models using the neutral particle transport code ARDRA \cite{hanebutte_ardra_1999}.

For alpha-eigenvalue problems, previous methods were limited to transport-based matrix methods or criticality search iterative schemes. For transport-based matrix methods, the discretized matrices of the alpha-eigenvalue were formed and traditional eigenvalue solution methods were applied. These methods become increasingly expensive in memory and computational effort as problems increase in complexity. For criticality search iterative schemes, the alpha-eigenvalue is determined by relating the eigenvalue to another eigenvalue problem, the $k$-effective eigenvalue. These methods require multiple $k$-effective eigenvalue calculations to converge the alpha-eigenvalue, increasing the number of transport sweeps required to converge the alpha-eigenvalue. A substantial number of iterations are therefore spent determining an eigenvalue not of interest to the problem. Furthermore, there remain open questions as to how converged the proxy eigenvalue calculation must be to allow for convergence of the alpha-eigenvalue problem. These methods also historically have been applied to supercritical nuclear systems, being of limited use for subcritical problems that are becoming more of interest in  the recent years. With growing interest in subcritical accelerator-driven reactor systems and subcritical experiment design, a numerical method that is able to deal with subcritical systems is required. The Rayleigh Quotient Fixed Point for alpha-eigenvalue problems is an iterative method that is able to solve subcritical. critical, and supercritical systems, providing a more general solution method to alpha-eigenvalue problems. The Rayleigh Quotient Fixed Point method directly updates the alpha-eigenvalue, not requiring knowledge of any other eigenvalue, substantially reducing the number of iterations required for convergence. For a wide variety of problems, the RQFP method for alpha-eigenvalue problems reduced the number of iterations required by up to factors of 50. It is also shown that the method uses an eigenvalue update that is optimal in the least squares sense.

For $k$-effective problems, the traditional workhorse method, the power method, traditionally used a fission source update for the eigenvalue at each iteration. By using the Rayleigh Quotient Fixed Point method, the $k$-effective eigenvalue problem can be viewed as a non-linear fixed-point method where an optimal update for the eigenvalue can be derived. In certain circumstances, such as infinite-medium problems used to determine homogenized cross section or problems where the angular flux eigenvector is highly localized, this provides a reduction in transport sweeps necessary to converge the eigenvalue and eigenvector of interest.

Throughout the derivation of the Rayleigh Quotient Fixed Point method, the primitivity property of the discretized eigenvalue equations was used to guarantee the existence of a unique positive eigenvector corresponding to the spectral radius as stated in the Perron-Froebenius theory for primitive matrices. It was shown that for a one-dimensional slab geometry eigenvalue problem discretized with diamond differencing in space, discrete ordinates angular quadrature, and the multigroup-in-energy approximation, the discretized linear system is a primitive system with index of primitivity of two. This result is similar to that of Mokhtar-Kharroubi, where it was found that the continuous neutron transport equation eigenvalue problem was \textit{positivity-improving}, or primitive with index of primitivity of two \cite{mokhtar1997mathematical}. The existence of a unique eigenvector and a way to relate it to either the alpha- or $k$-effective eigenvalue provided a powerful tool to derive a fixed-point method capable of determining the eigenvalue and the physical, positive angular flux eigenvector.

The Rayleigh Quotient Fixed Point method for alpha- and $k$-effective eigenvalues accurately determined the alpha- and $k$-effective eigenvalues of various analytic, infinite-medium problems by Betzler \cite{Betzler2014Alpha}. The failure of the method was also demonstrated in certain circumstances such as when the conditions of irreducibility and primitivity failed to exist. For these failure cases, it was shown that these problems were unphysical in most cases. Next, the alpha-eigenvalue RQFP method was validated for one-dimensional slabs and spheres and compared to the Green's Function Method of Kornreich and Parsons \cite{kornreich_timeeigenvalue_2005}. The number of transport sweeps required by the Rayleigh Quotient Fixed Point method was compared to the standard method, critical search. It was found that the RQFP method provided major reductions in the number of iterations required for convergence and provided the ability to converge problems where other methods failed. Next, realistic two-dimensional cylindrical problems and two- and three-dimensional fuel assembly benchmark problems were analyzed to show the general applicability of the RQFP method  even when some of the assumptions made in deriving the method did not apply. In these problems, the alpha-eigenvalue RQFP method provided substantial reductions in transport sweeps for realistic reactor problems with large amounts of materials, energies, and other heterogeneities.

For $k$-effective problems, the RQFP method provided moderate reductions of transport sweeps required for convergence. In particular, the RQFP method did well for infinite-medium problems. For one-dimensional slab and sphere problems, the $k$-effective eigenvalue RQFP method was competitive with the traditional power method with fission norm update method. For realistic two- and three-dimensional reactor problems, the RQFP method for $k$-effective eigenvalue problems provides another solution method that is easily implementable in neutron transport codes.

Acceleration of the Rayleigh Quotient Fixed Point method was shown to be possible using Anderson acceleration \cite{walker_anderson_2011}. For slowly converging alpha-eigenvalue problems solved using the RQFP method, Anderson acceleration provided acceleration of the linear fixed-point method convergence by a factor of up to ten. These slowly converging alpha-eigenvalue problems were characterized by large amounts of scattering and long critical widths. Neutrons in these systems would experience a large amount of scattering before finally being absorbed or leaked out of the system, slowing down convergence of the method.

By using a number of residual vectors, Anderson acceleration solves a constrained least squares problem for a set of weights used to generate a new vector iterate from a combination of residual vectors and the previous vector iterate. From a basic analysis, it was shown that each iteration of the Anderson acceleration method applied to the RQFP method was not much more expensive than one iteration of the RQFP method. However, the acceleration comes at the cost of increased memory.  The number of residual vectors used in the acceleration scheme impacts the convergence of the method, with more residual vectors decreasing the number of iterations required for convergence. However, in practice, each additional residual vector requires the same size of memory as the eigenvector of interest, with the memory cost quickly becoming prohibitive for large problems. This can be mitigated through the use of initial fixed-point iterations or relaxation coefficients which can help to increase the likelihood of the acceleration method converging to the correct eigenvalue/eigenvector pair. This increased memory cost requires a careful balancing of method parameters for each alpha-eigenvalue problem of interest. However, Anderson acceleration provides a useful option for the acceleration of the Rayleigh Quotient Fixed Point method.

The Rayleigh Quotient Fixed Point method for alpha- and $k$-effective eigenvalue problems reformulates two standard eigenvalue problems in nuclear engineering into non-linear fixed-point methods. The existence of a positive angular flux eigenvector is guaranteed by showing that in specific circumstances, the discretized eigenvalue problems form primitive systems of linear equations. For the alpha-eigenvalue problem, the RQFP method is capable of solving subcritical, critical, and supercritical problems for all types of problems of interest in nuclear engineering. In particular, the RQFP method is capable of converging subcritical systems that the traditional method, the critical search method, fails to solve. In comparison to the critical search method, the RQFP method provides a robust method that provides substantial reductions in the number of transport sweeps required for convergence, no matter the criticality of the system. For $k$-effective problems, the RQFP method provides another way to look at the eigenvalue problem. The method is competitive for various systems of interest when compared to the power method and excels in specific circumstances, such as infinite-medium problems. By looking at the linear algebraic structure of the discretized neutron transport eigenvalue problems, a method that guarantees the existence of the positive angular flux eigenvector and its corresponding eigenvalue was derived and shown to be robust, easily implementable in neutron transport codes, and an efficient solution method for eigenvalue problems of interest to nuclear engineers.

Future work required involves the proof of primitivity for two- and three-dimensional Cartesian geometry problems. In these problems, approximation of spatial derivatives by diamond differencing using a sufficiently small cell width no longer guarantees positivity of the angular flux solution \cite{greenbaum_iterative_1997}. Instead, step differencing is required in space. Given this limitation, it is sought to prove that the two- and three-dimensional Cartesian geometry discretized alpha- and $k$-effective eigenvalue equations form primitive systems. Another avenue for possible future work is the determination of the asymptotic constant coefficient describing the rate of converge of the system. It is thought that the asymptotic constant coefficient is related to the scattering cross section of the system but this has not been determined rigorously. Continued investigation into the use of Anderson acceleration is required to determine the optimal number of residual vectors given memory limitations and calculation costs. Given this acceleration method, slowly converging problems can be solved in fewer iterations allowing for the Rayleigh Quotient Fixed Point method to be widely applicable in all problems of interest.