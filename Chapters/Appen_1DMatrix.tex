\chapter[Discretization of the Alpha-Eigenvalue Problem For Slab Geometry][Discretized Slab Alpha-Eigenvalue Problem]{Discretization of the Alpha-Eigenvalue Problem For Slab Geometry}

\label{Discrete1D}

In Appendix~\ref{Discrete1D}, we describe the discretization of the alpha-eigenvalue problem for one-dimensional slab geometry using a matrix formalism similar to that of three-dimensional Cartesian geometry. In one dimension, the discretization of the continuous eigenvalue equation is substantially simpler and it may be helpful to the reader to study this case before analyzing the three-dimensional problem.

We begin with the alpha-eigenvalue neutron transport equation in one-dimensional slab geometry with isotropic scattering. The spatial domain is the interval $[a,b]$ in $x$, $\mu$ is the angle cosine in $[-1,1]$, the energy variable is $E \in [0, \infty)$, and the equations for the angular flux $\psi(x, \mu, E)$ are given by
\begin{multline}
\bigg [ \mu \frac{\partial}{\partial x} + \frac{\alpha}{v(E)} + \sigma(x,E) \bigg ] \psi(x,\mu,E) \\ = \chi(E) \int_{0}^{\infty} \diff E' \nu(E') \sigma_{f}(x,E') \int_{-1}^{1} \diff \mu' \psi(x,\mu',E) \\ + \int_{0}^{\infty} \diff E' \sigma_{s}(x, E' \rightarrow E) \int_{-1}^{1} \diff \mu' \psi(x,\mu',E').
\label{eq:1DAlpha}
\end{multline}
We assume vacuum Dirichlet conditions
\begin{align}
	\psi(a, \mu, E) &=0, \quad 0 < \mu \leq 1, \\
        \psi(b, \mu, E) &=0, \quad -1 \leq \mu < 0.
\end{align}
The discretization of Eq.~\ref{eq:1DAlpha} is done using diamond differencing in space, multigroup-in-energy, and discrete ordinates collocation in angle.

\section{Discretization of the One-Dimensional Slab Geometry Problem}

We begin by discretizing Eq.~\ref{eq:1DAlpha} in energy using the \textit{multigroup} approximation. We restrict the energy $E$ to a finite interval and partition the interval into groups:
\begin{equation}
	E_{max} = E_{0} > E_{1} > \dots > E_{G} = E_{min}.
\end{equation}
The eigenvalue equation is then averaged over each group $E_{g} < E < E_{g-1}$ and the cross sections are approximated by a flux-weighted average over each energy group. In the spatial dimension, we introduce a spatial grid
\begin{equation}
	a \equiv x_{0} < \dots < x_{i+1} < x_{i} < \dots < x_{M} \equiv b,
\end{equation}
and let 
\begin{equation}
\Delta x_{i} = x_{i} - x_{i-1}.
\end{equation} 
We refer to the $x_{i}$ as nodes and function values at the nodes are called nodal values. We assume that $\sigma_{g}$, $\sigma_{s,g,g'}$, and $\nu\sigma_{f,g}$, the total, scattering, and fission cross sections for energy group $g$, are constant on the zone $x_{i-1} < x < x_{i}$ and denote these values by $\sigma_{g,i}$, $\sigma_{s,g,g',i}$ and $\nu\sigma_{f,g,i}$. We use a discrete ordinates collocation of Eq.~\ref{eq:1DAlpha} at an even number of Gauss points $\mu_{\ell}$ with
\begin{equation}
	-1 < \mu_{1} < \dots < \mu_{L/2} < 0 < \mu_{L/2+1} < \dots < \mu_{L} < 1, \mu_{L+1-\ell} = - \mu_{\ell}.
\end{equation}
The integrals in angle in Eq.~\ref{eq:1DAlpha} are then approximated by
\begin{equation}
	\frac{1}{2} \int_{-1}^{1} \diff \mu \, \psi_{g}(x, \mu) \approx \sum_{\ell=1}^{L} w_{\ell} \psi_{g}(x, \mu_{\ell}).
\end{equation}
Using diamond differencing in the spatial dimension \cite{lewis_computational_1984}, we obtain the fully discretized set of equations for the eigenvalue problems
\begin{multline}
	\mu_{\ell} \frac{ \psi_{g,\ell,i} - \psi_{g, \ell, i-1}}{\Delta x_{i}} + \frac{\alpha}{v_{g}} \frac{\psi_{g,\ell,i} + \psi_{g, \ell, i-1}}{2} + \sigma_{g,i} \frac{\psi_{g,\ell,i} + \psi_{g, \ell, i-1}}{2} \\ = \frac{\chi_{g}}{2} \sum_{g'=1}^{G} \frac{\nu\sigma_{f,g',i}}{2} \sum_{\ell' = 1}^{L} w_{\ell'} \bigg ( \frac{\psi_{g',\ell',i} + \psi_{g',\ell',i-1} }{2} \bigg ) + \sum_{g'=1}^{G} \frac{\sigma_{s,g,g',i}}{2} \sum_{\ell' = 1}^{L} w_{\ell'} \bigg ( \frac{\psi_{g',\ell',i} + \psi_{g',\ell',i-1} }{2} \bigg ),
\label{eq:AlphaSlab}
\end{multline}
for $g = 1, \dots, G$, $i = 1, \dots, M$, and $\ell = 1, \dots, L$. The discretized boundary conditions are given by
\begin{align}
\psi_{g,\ell,M} &= 0 \text{ for } \ell = 1, \dots, L/2,  \\
\psi_{g,\ell,0} &= 0 \text{ for } \ell = L/2+1, \dots, L.
\end{align}


Using cell-centered flux values, it follows that Eq.~\ref{eq:AlphaSlab} is a system of $GL(M+1)$ equations for $GL(M+1)$ unknowns. 

%\subsection{The Diamond Difference Operator Matrix Form}

To write Eq.~\ref{eq:AlphaSlab} in matrix form, we define the angular flux vector for a single energy group $g$ as
\begin{equation}
\Psi_{g} \equiv 
\begin{pmatrix}
\Psi_{g,1} \\
\vdots \\
\Psi_{g,L}
\end{pmatrix} \in \mathbb{R}^{L(M+1)} \quad \text{ with } \quad
\Psi_{g, \ell} \equiv 
\begin{pmatrix}
\psi_{g,\ell,0} \\
\vdots \\
\psi_{g,L,M}
\end{pmatrix} \in \mathbb{R}^{M+1}.
\end{equation}
To write the matrix form of the diamond difference discretized operator $\mu_{\ell} \partial/\partial x + 1/v_{g} + \sigma_{g}$, we define the block diagonal matrix
\begin{equation}
\bar{S} \equiv \text{diag}(S_{1}, \dots, S_{L}) \in \mathbb{R}^{LM \times L(M+1)}
\end{equation}
with
\begin{equation}
S_{\ell} = S = \frac{1}{2}
\setlength\arraycolsep{2pt}
\begin{pmatrix}
1 & 1 & & \\
& \ddots & \ddots & \\
& & 1 & 1
\end{pmatrix} \in \mathbb{R}^{M \times (M+1)},
\end{equation}
for all $\ell$. The matrix $S$ interpolates nodal vectors into zone-centered vectors by averaging the nodal values. Now we define the total cross section and inverse velocity matrices for energy group $g$ as
\begin{equation}
	\Sigma_{g}  \equiv \text{diag}(\sigma_{g,1},\dots,\sigma_{g,M}) \in \mathbb{R}^{M \times M},
\end{equation}
\begin{equation}
	V^{-1}_{g}  \equiv \text{diag}(1/v_{g,1},\dots,1/v_{g,M}) \in \mathbb{R}^{M \times M}.
\end{equation}
We define the following matrices to describe the discretized derivative term
\begin{equation}
	\Delta x \equiv \text{diag}(\Delta x_{1}, \dots, \Delta x_{M}) \in \mathbb{R}^{M \times M}
\end{equation}
and
\begin{equation}
D \equiv
\setlength\arraycolsep{2pt}
\begin{pmatrix}
-1 & 1 & & \\
& \ddots & \ddots & \\
& & -1 & 1
\end{pmatrix} \in \mathbb{R}^{M \times (M+1)}.
\end{equation}
Boundary values are isolated by defining the row vector
\begin{equation}
	B_{\ell} \equiv \begin{cases}
				e_{M}^{T} \quad \text{ if } \ell \leq L/2, \\
				e_{0}^{T}  \quad \text{ if } \ell > L/2,
                              \end{cases} \in \mathbb{R}^{M+1},
\end{equation}
where the indices on the standard basis vectors $e_{\ell}$ are from 0 to $M$. Finally, we define the matrices $Z$ and $Z_{b}$ as
\begin{equation}
	Z \equiv \begin{pmatrix}
			I_{M} \\
			0
		     \end{pmatrix} \in \mathbb{R}^{(M+1) \times M} \quad \text{ and } \quad
	Z_{b} \equiv  e_{M}.
\end{equation}
We can now define the matrix form of the diamond difference representation of  $\mu_{\ell} \partial/\partial x + \alpha/v_{g} + \sigma_{g}$ as 
\begin{equation}
	H_{g} + \alpha V^{-1}_{g} \equiv \text{diag}(H_{g,1}, \dots, H_{g,L}) + \alpha \text{ diag}(V^{-1}_{g,1}, \dots, V^{-1}_{g,L}) \in \mathbf{R}^{L(M+1)},
\end{equation}
where
\begin{equation}
	H_{g,\ell} + \alpha V^{-1}_{g,\ell} \equiv Z(\mu_{\ell}\Delta x^{-1}D + \Sigma_{g}S_{\ell}) + Z_{b}B_{\ell} + \alpha ZV_{g}^{-1}S_{\ell}.
\end{equation}
It can be shown that $H_{g} + \alpha V^{-1}_{g}$ is nonsingular for the diamond difference method if $\alpha$ is not too negative \cite{greenbaum_iterative_1997}.

We now define discretized representations of angular flux moment operators. The matrices operate on zone-centered vectors and are in $\mathbb{R}^{M \times LM}$. We define the matrix
\begin{equation}
	L_{n} \equiv (l_{n}W) \otimes I_{M},
\end{equation}
where $l_{n} \equiv (P_{n}(\mu_{1}), P_{n}(\mu_{2}), \dots, P_{n}(\mu_{L}))$ are the Legendre polynomials and the quadrature weights are given by $W \equiv \text{diag}(w_{1}, \dots, w_{L})$.
If the vector $\Psi_{g}$ approximates $\psi_{g}(x, \mu)$, then $L_{n}\Psi_{g}$ approximates taking the n$^{th}$ moment of the angular flux $\phi_{g,n}(x)$. We also define the matrix
\begin{equation}
	L_{n}^{+} \equiv (2n+1)l_{n}^{T} \otimes I_{M} \in \mathbb{R}^{LM \times M}.
\end{equation}
If a vector $\Phi$ approximates $\phi(x)$, then $L_{n}^{+}\Psi$ will approximate $P_{n}(\mu)\phi(x)$. We define the grouped matrices for $N_{s}$ moments as
\begin{equation}
L^{N} = \begin{pmatrix}
		L_{0} \\
		\vdots \\
		L_{N}
	     \end{pmatrix} \quad \text{ and } \quad
L^{N,+} = (L_{0}^{+}, \dots, L_{N}^{+}).
\end{equation}
We can define the scattering and fission matrices as
\begin{equation}
	\Sigma_{s,g,g',n} \equiv \text{diag}(\sigma_{s,g,g',n,1}, \dots, \sigma_{s,g,g',n,M}) \in \mathbb{R}^{M \times M}
\end{equation}
and
\begin{equation}
	\Sigma_{f,g,g',n} \equiv \text{diag}(\chi_{g}\nu\sigma_{f,g',n,1}, \dots, \chi_{g}\nu\sigma_{f,g',n,M}) \in \mathbb{R}^{M \times M}.
\end{equation}
We now define matrices that inject zone-centered vectors into nodal vector space and vice versa. We define the matrices
\begin{equation}
\bar{\Sigma}_{g} \equiv I_{L} \otimes \Sigma_{g} \in \mathbb{R}^{LM \times LM},
\end{equation}
\begin{equation}
\bar{V^{-1}}_{g} \equiv I_{L} \otimes V^{-1}_{g} \in \mathbb{R}^{LM \times LM},
\end{equation}
\begin{equation}
	\bar{Z} = I_{L} \otimes Z \in \mathbb{R}^{L(M+1) \times LM},
\end{equation}
\begin{equation}
	\bar{Z}_{B} = I_{L} \otimes Z_{b} \in \mathbb{R}^{L(M+1) \times L},
\end{equation}
\begin{equation}
	B = \text{diag}(B_{1}, \dots, B_{L}) \in \mathbb{R}^{L \times L(M+1)},
\end{equation}
and
\begin{equation}
	C = \text{diag}(\mu_{1}\Delta x^{-1}D, \dots, \mu_{L} \Delta x^{-1}D) \in \mathbb{R}^{LM \times L(M+1)}.
\end{equation}
Using the above matrices, we can write the matrix $H_{g} + V^{-1}_{g}$ as
\begin{multline}
H_{g} + V^{-1}_{g} \equiv \text{diag}(H_{g,1}, \dots, H_{g,L}) + \text{diag}(V^{-1}_{g,1}, \dots, V^{-1}_{g,L}) \\ = \bar{Z}(C + \bar{\Sigma}_{g}\bar{S}) + \bar{Z}_{B}B + \bar{Z}\bar{V}^{-1}\bar{S}.
\end{multline}
The discretized multigroup eigenvalue equations can then be written in the matrix form as
\begin{equation}
	H_{g} \Psi_{g} + \alpha V^{-1}_{g}\Psi_{g} = \bar{Z} \sum_{g'=1}^{G} \sum_{n=0}^{N_{s}} L_{n}^{+}\Sigma_{s,g,g',n}L_{n}\bar{S}\Psi_{g'}  +  \bar{Z} \sum_{g'=1}^{G} \sum_{n=0}^{N_{s}} L_{n}^{+}\Sigma_{f,g,g',n}L_{n}\bar{S}\Psi_{g'}, 
	\label{eq:AlphaMGg}
\end{equation}
Finally, we can write the multigroup discretized eigenvalue equations if we define the matrices
\begin{equation}
	\mathbf{\Psi} \equiv \begin{pmatrix}
					\Psi_{1} \\
					\Psi_{2} \\
					\vdots \\
					\Psi_{G}
				       \end{pmatrix}, \quad
	\mathbf{\Sigma_{s}} \equiv \begin{pmatrix}
					\Sigma_{s, 11}^{N_{s}} & \dots & \Sigma_{s,1G}^{N_{s}} \\
					\vdots & \ddots & \vdots \\
					\Sigma_{s, G1}^{N_{s}} & \dots & \Sigma_{s,GG}^{N_{s}}
					\end{pmatrix}, \quad 
	\mathbf{\Sigma_{f}} \equiv \begin{pmatrix}
					\Sigma_{f, 11}^{N_{s}} & \dots & \Sigma_{f,1G}^{N_{s}} \\
					\vdots & \ddots & \vdots \\
					\Sigma_{f, G1}^{N_{s}} & \dots & \Sigma_{f,GG}^{N_{s}}
					\end{pmatrix},
\end{equation}
where 
\begin{equation}
\Sigma_{s,gg'}^{N_{s}} \equiv \text{diag}(\Sigma_{s,g,g',0}, \dots, \Sigma_{s,g,g',N_{s}})
\end{equation}
and
\begin{equation}
\Sigma_{f,gg'}^{N_{s}} \equiv \text{diag}(\Sigma_{f,g,g',0}, \dots, \Sigma_{f,g,g',N_{s}}).
\end{equation}
Defining the following matrices 
\begin{equation}
\mathbf{S} \equiv I_{G} \otimes \bar{S},
\end{equation} 
\begin{equation}
\mathbf{Z} \equiv I_{G} \otimes \bar{Z},
\end{equation}
\begin{equation}
\mathbf{H} + \mathbf{V}^{-1} \equiv \text{diag}(H_{1} + V^{-1}_{1}, H_{2} + V^{-1}_{2}, \dots, H_{G} + V^{-1}_{G}),
\end{equation}
\begin{equation}
\mathbf{L}^{+} \equiv I_{G} \otimes L^{N_{s},+},
\end{equation}
\begin{equation}
\mathbf{L} \equiv I_{G} \otimes L^{N_{s}},
\end{equation} 
then Eq.~\ref{eq:AlphaMGg} can be written as
\begin{equation}
	\big ( \mathbf{H} + \alpha \mathbf{V}^{-1} \big ) \mathbf{\Psi} = \mathbf{Z} \mathbf{L}^{+}  \big ( \mathbf{\Sigma_{s}} + \mathbf{\Sigma_{f}} \big )\mathbf{L} \mathbf{S} \mathbf{\Psi}.
	\label{eq:AlphaMG1D}
\end{equation}
Similarly, the discretized $k$-eigenvalue problem can be written as
\begin{equation}
	\mathbf{H}\mathbf{\Psi}  = \mathbf{Z} \mathbf{L}^{+} \bigg ( \mathbf{\Sigma_{s}} + \frac{1}{k} \mathbf{\Sigma_{f}} \bigg ) \mathbf{L} \mathbf{S} \mathbf{\Psi}.
	\label{eq:kMG1D}
\end{equation}

Equations~\ref{eq:AlphaMG1D} and \ref{eq:kMG1D} are eigenvalue equations for the criticality eigenvalue and the node-centered angular flux eigenvector. In the derivation of the Rayleigh Quotient Fixed Point method, an inner product is required. However, the inner product is defined for zone-centered vectors, whereas the unknown angular flux vectors in Eqs.~\ref{eq:AlphaMG1D} and \ref{eq:kMG1D} are node-centered. To satisfy this requirement, Eqs.~\ref{eq:AlphaMG1D} and \ref{eq:kMG1D} are rewritten using zone-centered angular flux eigenvectors. We denote $\mathbf{\Psi_{z}}$ as the zone-centered unknown and $\mathbf{H_{z}}$ as the zone-centered version of $\mathbf{H}$.

Given a zone-centered angular flux vector $\mathbf{\Psi_{z}}$, the nodal angular flux vector $\mathbf{\Psi}$ defined by
\begin{equation}
	\Psi_{g,\ell} \equiv (ZS + Z_{b}B_{\ell})^{-1}Z \Psi_{z,g,\ell} \text{ for all } g = 1, 2, \dots, G \text{ and } \ell = 1, 2, \dots, L,
\end{equation}
satisfies $B_{\ell} \Psi_{g,\ell} = 0$ and $S \Psi_{g,\ell} = \Psi_{z,g,\ell}$ for all $g$ and $\ell$. Defining the matrices
\begin{equation}
	\mathbf{C} \equiv I_{G} \otimes C,
\end{equation}
\begin{equation}
	\mathbf{B} \equiv I_{G} \otimes B,
\end{equation}
\begin{equation}
	\mathbf{Z_{B}} \equiv I_{G} \otimes \bar{Z}_{B},
\end{equation}
and
\begin{equation}
	\mathbf{\Sigma} \equiv \text{diag}(\bar{\Sigma}_{1}, \bar{\Sigma}_{2}, \dots, \bar{\Sigma}_{G}),
\end{equation}
then $\mathbf{H}$ and $\mathbf{V}^{-1}$ can be rewritten as
\begin{equation}
	\mathbf{H} = \mathbf{Z}(\mathbf{C} + \mathbf{\Sigma S}) + \mathbf{Z_{B} B},
\end{equation}
and
\begin{equation}
	\mathbf{V}^{-1} = \mathbf{Z}\mathbf{V}^{-1} \mathbf{S}.
\end{equation}
For the nodal-centered angular flux $\mathbf{\Psi}$, we have from Lemma~\ref{lemma:Hz} that
\begin{equation}
	\mathbf{\Psi} = (\mathbf{ZS} + \mathbf{Z_{B}B})^{-1}\mathbf{Z}\mathbf{\Psi_{z}}.
	\label{eq:PsiPsiz1D}
\end{equation}
Substituting Eq.~\ref{eq:PsiPsiz1D} into Eq.~\ref{eq:AlphaMG1D} and \ref{eq:kMG1D} and multiplying on the left by $\mathbf{Z}^{T}$ gives
\begin{equation}
	\boxed{\big ( \alpha \mathbf{V}_\mathbf{z}^{-1} + \mathbf{H_{z}} \big ) \mathbf{\Psi_{z}} = \mathbf{L}^{+}( \mathbf{\Sigma_{s}} + \mathbf{\Sigma_{f}}) \mathbf{L} \mathbf{\Psi_{z}},}
	\label{eq:AlphaZ1D}
\end{equation}
where
\begin{equation}
	\mathbf{H_{z}} \equiv \mathbf{C}(\mathbf{ZS} + \mathbf{Z_{B}B})^{-1}\mathbf{Z} + \mathbf{\Sigma},
\end{equation}
and
\begin{equation}
	\mathbf{V}_{\mathbf{z}}^{-1} \equiv \mathbf{V}^{-1} \mathbf{S} (\mathbf{ZS} + \mathbf{Z_{B}B})^{-1}\mathbf{Z}.
\end{equation}

Following the same procedure for the $k$-effective eigenvalue neutron transport equation yields the discretized equation:
\begin{equation}
	\boxed{\mathbf{H_{z}} \mathbf{\Psi_{z}} = \mathbf{L}^{+} \bigg ( \mathbf{\Sigma_{s}} + \frac{1}{k}\mathbf{\Sigma_{f}} \bigg ) \mathbf{L} \mathbf{\Psi_{z}}.}
	\label{eq:kZ1D}
\end{equation}

Equations~\ref{eq:AlphaZ1D} and \ref{eq:kZ1D} are the cell-centered eigenvalue equations for one-dimensional slab geometry.