% (This file is included by thesis.tex; you do not latex it by itself.)

\begin{abstract}

% The text of the abstract goes here.  If you need to use a \section
% command you will need to use \section*, \subsection*, etc. so that
% you don't get any numbering.  You probably won't be using any of
% these commands in the abstract anyway.

The alpha- and k-effective eigenproblems describe the criticality and fundamental neutron flux mode of a nuclear system. Traditionally, the alpha-eigenvalue problem has been solved using methods that focus on supercritical systems with large, positive eigenvalues. These methods, however, struggle for very subcritical problems where the negative eigenvalue can lead to negative absorption, potentially causing the methods to diverge. The k-effective eigenvalue problem has generally been solved using power iteration. For problems with dominance ratios close to one, however, power iteration is slow to converge and requires acceleration. We present the Rayleigh Quotient Fixed Point methods, nonlinear fixed-point methods, that are applied to the primitive discretizations of the neutron transport eigenvalue equations. We prove that the discretized eigenvalue equations form a primitive linear system for one-dimensional slab geometry where the unique, positive angular flux eigenvectors are guaranteed to exist from the Froebenius-Perron Theorem for primitive matrices. These methods are capable of solving subcritical and supercritical alpha- and k-effective eigenvalue problems. The derived eigenvalue updates are proven to be optimal in the least squares sense and positive eigenvector updates are guaranteed from the properties of primitive matrices. We consider infinite-medium, one-dimensional slabs and spheres, two-dimensional cylinders, and three-dimensional quarter core benchmark problems and show the ability of the Rayleigh Quotient Fixed Point method to obtain the fundamental eigenvalue and eigenvector of these systems, even when the discretized eigenvalue equations no longer form a primitive system. We also consider the use of Anderson acceleration to accelerate the convergence of the Rayleigh Quotient Fixed Point methods.

\end{abstract}
