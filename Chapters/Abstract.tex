% (This file is included by thesis.tex; you do not latex it by itself.)

\begin{abstract}

% The text of the abstract goes here.  If you need to use a \section
% command you will need to use \section*, \subsection*, etc. so that
% you don't get any numbering.  You probably won't be using any of
% these commands in the abstract anyway.

The alpha- and k-effective eigenproblems describe the criticality and fundamental neutron flux mode of a nuclear system. Traditionally, the alpha-eigenvalue problem has been solved using methods that focus on supercritical systems with large, positive eigenvalues. These methods, however, struggle for very subcritical problems where the negative eigenvalue can lead to negative absorption, potentially causing the methods to diverge. The k-effective eigenvalue problem has generally been solved using power iteration. For problems with dominance ratios close to one, however, power iteration can be intractably slow. 

We present the Rayleigh Quotient Fixed Point (RQFP) methods, nonlinear fixed-point methods that are applied to the primitive discretizations of the neutron transport eigenvalue equations. We prove that the discretized eigenvalue equations form a primitive linear system for one-dimensional slab geometry where the unique, positive angular flux eigenvectors are guaranteed to exist from the Perron-Frobenius Theorem for primitive matrices. These methods are capable of solving subcritical and supercritical alpha- and k-effective eigenvalue problems. The derived eigenvalue updates are proven to be optimal in the least squares sense and positive eigenvector updates are guaranteed from the properties of primitive matrices. We consider infinite-medium, one-dimensional slabs and spheres, two-dimensional cylinders, and three-dimensional quarter core benchmark problems and show the ability of the Rayleigh Quotient Fixed Point method to obtain the fundamental eigenvalue and eigenvector of these systems, even when the discretized eigenvalue equations no longer form a primitive system. We also consider the use of Anderson acceleration to accelerate the convergence of the Rayleigh Quotient Fixed Point methods.

We demonstrate that for alpha-eigenvalue problems, the Rayleigh Quotient Fixed Point method substantially reduces the number of iterations required for convergence when compared to traditional alpha-eigenvalue methods such as the critical search method. For a wide variety of problems, the RQFP method for alpha-eigenvalue problems reduces the number of iterations required by up to factors of 50 and converges problems that other methods are unable to converge. For $k$-effective problems, the RQFP method provides moderate reductions of iterations required for convergence when compared to power iteration. In particular, the RQFP method does well for infinite-medium problems or problems where the eigenvector is highly localized. We also demonstrate acceleration of the RQFP method by Anderson acceleration. For slowly converging alpha-eigenvalue problems solved using the RQFP method, Anderson acceleration can provide acceleration of the linear fixed-point method convergence by a factor of up to ten.

By looking at the linear algebraic structure of the discretized neutron transport eigenvalue problems, the RQFP method guarantees the existence of the positive angular flux eigenvector and its corresponding eigenvalue. By examining a wide variety of problems of interest to nuclear engineers, we show that the RQFP method is robust, easily implementable in neutron transport codes, and an efficient solution method for eigenvalue problems in neutron transport.
\end{abstract}
