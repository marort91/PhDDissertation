\chapter{Introduction}

The discovery of nuclear fission in 1939 by Otto Hahn precipitated a revolution in science and politics. The construction of the first experimental nuclear reactor and the detonation of the world's first nuclear weapon forced societies throughout the world to grapple with this newfound energy source, whose use could power societies for centuries or bring ruin to its cities within hours. As countries throughout the world built nuclear reactors for energy, the nuclear weapon states built up nuclear arsenals of unfathomable destructive power. With the ever looming threat of nuclear warfare and nuclear reactor accidents in the latter half of the twentieth century, nuclear energy became feared, despite its utility in medicine and electric power production.  Nuclear energy and nuclear technologies remain controversial to this day. Fear of radiation (perhaps irrational in certain circumstances), nuclear weapon proliferation (a valid concern given the proliferation of nuclear weapons in the past three decades), and the open question of what to do with nuclear waste (a question of both engineering and policy), have slowed the spread of peaceful nuclear technologies. However, as the demand for clean, sustainable energy increases throughout the world and global climate change threatens societies, nuclear energy is once again poised to deliver the answer to an energy hungry world. Soothing the fears of nuclear energy requires forward thinking, where science and engineering come together with policy to create a culture of safety, risk management, and certainty. 

To this end, advances in nuclear engineering technologies have given scientists and engineers the tools necessary to solve the problems of nuclear energy and design even safer, more sustainable nuclear technologies. Whereas seventy years ago the design of nuclear systems required much approximation due to the limited knowledge of neutron transport and the nature of computation, today, designers take advantage of larger computing power and memory to model increasingly complex designs and problems with less approximation. No longer limited by memory, the complex nature of neutron interactions as modeled by material cross sections can be more fully captured. No longer limited by computation, high-fidelity models taking into account complex geometric designs and complex nuclear cross section data can be solved for large numbers of unknowns, giving insight into the time-behavior of materials, reaction rates, and energy production. Continued advances in the mathematical field of neutron transport have allowed for the design of efficient solution algorithms, taking complete advantage of computing power. 

However, despite the progress made in the last few decades in the modeling and design of nuclear systems, much remains to be done. With increasingly complex reactor and accelerator designs, the modeling and numerical techniques of before may no longer work the best or work at all. With increasing computational resources, the problems designers seek to solve have scaled along with them. Once again, we find ourselves stretching the ability of our computational knowledge and our methods. Now, algorithms are required that efficiently scale and make use of the computational resources available. Mathematical methods are required whose properties are known and understood for systems consisting of thousand of materials and irregular and unique geometries. The problems that must be solved are only limited by the imagination. That same sort of imagination is now required for the design of efficient solution algorithms capable of taking complete advantage of our newfound computing power. It is here that this dissertation seeks to make a new contribution to the field. 

This dissertation presents a new method, the Rayleigh Quotient Fixed Point (RQFP) method, to solve the the alpha- and $k$-effective eigenproblems of neutron transport. These eigenproblems describe the criticality and fundamental neutron flux mode of nuclear systems. Using a Rayleigh quotient minimization principle that is applied to demonstrably primitive discretizations of the neutron transport eigenvalue problems and the properties of primitive matrices, a new iterative method is derived. The derived eigenvalue updates are optimal in the least squares sense and positive eigenvector updates are guaranteed from the Froebenius-Perron Theorem for primitive matrices \cite{birkhoff_reactor_1958}. For alpha-eigenvalue problems, whereas traditional techniques have focused on supercritical problems and were limited in subcritical cases \cite{hill_efficient_1983}, this method allows for the solution of both subcritical and supercritical systems. For $k$-effective eigenvalue problems, traditionally the update for the eigenvalue has been taken to be some norm of the angular flux. In particular, the total fission rate over the domain is often used to update $k$. It has been observed that using the Rayleigh quotient can improve the efficiency of the power method \cite{warsa2004krylov}. We show this is due to the eigenvalue update being optimal in the least squares sense. 
We discuss the development, applicability, strengths and weaknesses of the method in this dissertation. The dissertation proceeds as follows

\begin{itemize}
	\item Chapter 2 discusses neutron transport, the criticality eigenvalue problems of neutron transport, the methods used to solve these problems, as well as reviewing linear algebra and fixed-point concepts.
	\item Chapter 3 discusses the discretization of the continuous criticality eigenvalue problems into matrix equations. These matrix equations are shown to be primitive matrices.
	\item Chapter 4 derives the Rayleigh Quotient Fixed Point method for alpha- and $k$-effective eigenvalue problems. Using the properties of primitive matrices, a fixed-point method is derived to determine the eigenvalue and eigenvector of the system. The Jacobians of the non-linear fixed-point methods are derived and their implication regarding the convergence of the methods discussed.
	\item Chapter 5 discusses the performance of the Rayleigh Quotient Fixed Point method for infinite medium problems. Both one-speed and multigroup benchmarks and analytical test problems are used to demonstrate the correctness of the method and benchmark its performance as compared to other standard eigenproblem methods used in the field. The chapter also discusses in which circumstances the method might fail to converge.
	\item Chapter 6 discusses the performance of the method for one-dimensional slab and spherical geometries. Benchmark and analytical test problems are used to show the wide applicability of the method for various realistic cross section sets, one-speed and multigroup-in-energy problems, and heterogenous media.
	\item Chapter 7 discusses the performance of the Rayleigh Quotient Fixed Point method for two- and three-dimensional Cartesian geometry and two-dimensional cylindrical geometry. Two- and three-dimensional fuel pin and fuel assembly problems are considered and the method compared to standard nuclear engineering eigenproblem methods.
	\item Chapter 8 discusses the use of Anderson Acceleration to accelerate the Rayleigh Quotient Fixed Point method for alpha-eigenvalue problems. We discuss the benefits and costs for using acceleration in the context of neutron transport.
	\item Chapter 9 summarizes and reviews the Rayleigh Quotient Fixed Point method for criticality eigenvalue problems and discusses future work.
	\item Appendix A discusses the discretization of the one-dimensional slab alpha-eigenvalue neutron transport equation.
	\item Appendix B contains the MATLAB implementation of Anderson Acceleration discussed in Chapter 8.
\end{itemize}

We find that the Rayleigh Quotient Fixed Point method allows for the solution of subcritical, critical, and supercritical alpha-eigenvalue problems for all types of nuclear systems of interest. The RQFP method solves problems where traditional method fail to converge, specifically subcritical systems and systems without fissile material. The RQFP method also drastically reduces the number of iterations required to converge the solution when compared to traditional alpha-eigenvalue methods. The RQFP method applied to $k$-effective eigenvalue gives designers another solution method. In specific circumstances, like infinite-medium problems, the RQFP method for $k$-effective eigenvalue problems improves upon the performance of traditional eigensolvers used in nuclear engineering. We find that in specific circumstances, the alpha- and $k$-effective eigenvalue problems form primitive linear systems. By using the properties of primitive linear systems, the RQFP method provides a robust and easily implementable method for the solution of alpha- and $k$-effective eigenvalue problems of interest to the nuclear engineering community.