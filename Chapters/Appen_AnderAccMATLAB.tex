\chapter[Implementation of Anderson Acceleration in Matlab for the Alpha-Eigenvalue Rayleigh Quotient Fixed Point Method][Anderson Acceleration Matlab Implementation]{Implementation of Anderson Acceleration in Matlab for the Alpha-Eigenvalue Rayleigh Quotient Fixed Point Method}

\label{AnderAccMATLAB}

In Appendix~\ref{AnderAccMATLAB} we describe the implementation of Anderson acceleration for the Rayleigh Quotient Fixed Point method for alpha-eigenvalue problems. We include the Matlab code for future use. First, we describe the one-sweep alpha-eigenvalue Rayleigh Quotient Fixed Point implementation and its input and outputs. We then describe the Anderson acceleration implementation of the unconstrained problem and describe the various features of the method. We describe the solution process for the update and the matrix deletion required to maintain acceptable conditioning of the system.

\section{Alpha-Eigenvalue Rayleigh Quotient Fixed Point Method Matlab Implementation}

\label{sec:AlphaRQSweep}

The Matlab function \texttt{AlphaRQSweep} performs one iteration (transport sweep) of the Rayleigh Quotient Fixed Point method for alpha-eigenvalue problems. The function requires the matrices \texttt{H, S, F, iV}, the numerical matrix representations of the matrices $\mathbf{H}_{\mathbf{z}}$, $\mathbf{V}_{\mathbf{z}}^{-1}$, $\mathbf{\Sigma_{s}}$, and $\mathbf{\Sigma_{f}}$ described in Chapter~\ref{Discrete}. The function also requires input vectors \texttt{x} and \texttt{q}, the previous angular flux vector and previous source, respectively, as described in Chapter~\ref{DeriveRQ}. The function calculates the alpha-eigenvalue and returns it to \texttt{alpha} and determines the next angular flux iterate and source. \texttt{AlphaRQSweep} is used as the fixed point function evaluation in Anderson acceleration. The Matlab code is shown in Listing~\ref{AlphaRQCode}.

\clearpage

\lstinputlisting[numbers=none, label=AlphaRQCode, caption=\texttt{AlphaRQSweep}-Alpha-Eigenvalue RQFP Matlab Implementation, captionpos=b]{Matlab/AlphaRQSweep.m}

\clearpage

\section{Anderson Acceleration Matlab Implementation}

The Matlab function \texttt{AndersonAcc} performs the Anderson acceleration of the Rayleigh Quotient Fixed Point method for alpha-eigenvalue problems. The function requires the matrices \texttt{H, S, F, iV}, the number of maximum iterations, \texttt{maxits}, the $\ell_{2}$ norm tolerance, \texttt{tol}, the maximum number of residual vectors to be used in the method, \texttt{mmax}, the number of initial fixed-point iterations, \texttt{fpiters}, and the relaxation parameter, \texttt{beta}. These parameters are described in Chapter~\ref{sec:AndAcc}.

The function \texttt{AndersonAcc} initializes the initial angular flux guess and source and then performs a fixed number of initial fixed-point function evaluations given by the input argument \texttt{fpiters} before beginning the acceleration of the fixed-point method. The fixed-point evaluation is done by calling the function \texttt{AlphaRQSweep}, described in Section~\ref{sec:AlphaRQSweep}. The function then performs a maximum number of iterations given by \texttt{maxiters} or until the residual tolerance is less than the value given by the input argument \texttt{tol}. The function then performs one fixed-point function evaluation of \texttt{AlphaRQSweep}, calculated the residual and sets the residual vector matrix, deletes old residual vector columns, performs a QR decomposition, and then sets the new angular flux iterate. The function returns the converged alpha-eigenvalue, the converged angular flux vector, and a vector of residuals. The Matlab code is shown in Listing~\ref{AACode}.

\lstinputlisting[numbers=none, label=AACode, caption=\texttt{AndersonAcc}-Anderson Acceleration Matlab Implementation, captionpos=b]{Matlab/AndersonAcc.m}